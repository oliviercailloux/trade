\RequirePackage[l2tabu, orthodox]{nag}
\documentclass[pagesize, twoside=off, bibliography=totoc, DIV=12, fontsize=12pt, a4paper]{scrartcl}
\input{preamble/packages}
\input{preamble/redac}
%The “natural numbers”, ambiguous whether it includes zero; to be used only for “\in \N” to mean that a set is finite.
\NewDocumentCommand{\N}{}{ℕ}
%The “non-negative integers”, thus including zero.
\NewDocumentCommand{\Nz}{}{ℕ_0}
%Consensually, the “integers”, also known as the “whole numbers”.
\NewDocumentCommand{\Z}{}{ℤ}
%Reading the answers to https://math.stackexchange.com/questions/1220760/what-is-the-difference-between-natural-numbers-and-positive-integers/, I get that it seems (more or less) consensual that Z^+ = 1, … designates the “positive integers” (implying, strictly), also known as the “counting numbers”. 
\NewDocumentCommand{\Zp}{}{ℤ^+}
\NewDocumentCommand{\Q}{}{ℚ}
\NewDocumentCommand{\R}{}{ℝ}
%\mathscr is rounder than \mathcal.
\NewDocumentCommand{\powerset}{m}{\mathscr{P}(#1)}
%Powerset without zero (with p for “positive”).
\NewDocumentCommand{\powersetp}{m}{\mathscr{P}(#1) \setminus \set{\emptyset}}
%https://tex.stackexchange.com/a/45732, works within both \set and \set*, same spacing than \mid (https://tex.stackexchange.com/a/52905).
\NewDocumentCommand{\suchthat}{}{\;\ifnum\currentgrouptype=16 \middle\fi|\;}
\NewDocumentCommand{\st}{}{\text{ s.\ t.\ }}
\NewDocumentCommand{\knowing}{}{\;\ifnum\currentgrouptype=16 \middle\fi|\;}
%Integer interval.
\NewDocumentCommand{\intvl}{m}{⟦#1⟧}
%Allows for \abs and \abs*, which resizes the delimiters.
\DeclarePairedDelimiter\abs{\lvert}{\rvert}
\NewDocumentCommand\card{m}{\##1}
%\DeclarePairedDelimiter\card{\lvert}{\rvert}
\DeclarePairedDelimiter\floor{\lfloor}{\rfloor}
\DeclarePairedDelimiter\ceil{\lceil}{\rceil}
%Perhaps should use U+2016 ‖ DOUBLE VERTICAL LINE here?
\DeclarePairedDelimiter\norm{\lVert}{\rVert}
%From mathtools. Better than using the package braket because braket introduces possibly undesirable space. Then: \begin{equation}\set*{x \in \R^2 \suchthat \norm{x}<5}\end{equation}.
\DeclarePairedDelimiter\set{\{}{\}}
\DeclareMathOperator*{\argmax}{arg\,max}
\DeclareMathOperator*{\argmin}{arg\,min}
% Semantic relations: strict subset (included in and not equal) and weak subset (included in and possibly equal)
\NewDocumentCommand{\ssub}{}{\subsetneq}
\NewDocumentCommand{\wsub}{}{\subseteq}
% Thanks to https://math.stackexchange.com/questions/1121553/is-there-a-notation-for-being-a-finite-subset-of, only suitable when the right hand side is infinite (otherwise, need to specify whether the subsetting is strict).
\NewDocumentCommand{\fsub}{}{\subset_\text{fin}}

%UTR #25: Unicode support for mathematics recommend to use the straight form of phi (by default, given by \phi) rather than the curly one (by default, given by \varphi), and thus use \phi for the mathematical symbol and not \varphi. I however prefer the curly form because the straight form is too easy to mix up with the symbol for empty set.
\let\phi\varphi

%The amssymb solution.
\NewDocumentCommand{\restr}{mm}{{#1}_{\restriction #2}}
%Another acceptable solution.
%\NewDocumentCommand{\restr}{mm}{{#1|}_{#2}}
%https://tex.stackexchange.com/a/278631; drawback being that sometimes the text collides with the line below.
%\NewDocumentCommand\restr{mm}{#1\raisebox{-.5ex}{$|$}_{#2}}


\input{preamble/math_mine}
%\input{preamble/draw}
%\input{preamble/jdoc}

\title{International Economics II: course notes \thanks{Course given by Gabriel Smagghue}}
\author{Olivier Cailloux}
\author{Name2}
\affil{Université Paris-Dauphine, PSL Research University, CNRS, LAMSADE, 75016 PARIS, FRANCE\\
  \href{mailto:olivier.cailloux@dauphine.psl.eu}{olivier.cailloux@dauphine.psl.eu}
}
\author{Name3}
\affil{Affil2}
\hypersetup{
  pdfsubject={},
  pdfkeywords={},
}

\begin{document}
\maketitle

\section{K79}
\subsection{Set up}
\begin{itemize}
  \item $U(\bar{c}) = \sum_{1 ≤ i ≤ n} u(c_i)$, with $\bar{c} = \set{c_i}_{1 ≤ i ≤ n}$
  \item $u(0) = 0, u'(0) > 0, u''(0) < 0$
  \item Elasticity of demand of good $i$: $ε_i = \frac{∂c_i}{∂p_i} \frac{p_i}{c_i}$, we assume $ε_i < -1$
  \item Define $σ(c_i) = - ε_i$, thus $1 < σ(c_i)$
  \item Increasing Return to Scale (IRS): workload required for producing $q_i$ units is $l_i = f + q_i / φ$; $f$ the fixed cost, $φ$ the productivity
\end{itemize}

\subsection{Consumption}
\begin{itemize}
  \item From Lagrangien for utility maximization subject to revenue constraint, we obtain $u'(c_i) = λ p_i$
  \item $dλp_i + λdp_i = u''(c_i) dc_i$
  \item We assume $\frac{dλ}{dp_i} = 0$ (large number of varieties), whence $λ = u''(c_i) \frac{dc_i}{dp_i}$
  \item Thus, $u'(c_i) / p_i = u''(c_i) \frac{dc_i}{dp_i}$
  \item We obtain $σ(c_i) = - \frac{u'(c_i)}{u''(c_i) c_i} > 0$
\end{itemize}

\subsection{Production}
\begin{itemize}
  \item Cost of producing $q_i$ units is $w l_i$; $w$ the wage
  \item $L$ identical consumers thus $q_i = L c_i$; whence $σ(c_i) = - \frac{∂q_i}{∂p_i} \frac{p_i}{q_i}$
  \item Marginal Cost is $\textit{MC} = \frac{w}{φ}$
  \item Revenue of the firm on product $i$ is $R_i = p_i q_i$ (and $R = \sum_{1 ≤ i ≤ n} p_i q_i$)
  \item Marginal Revenue is $MR_i = \frac{dp_i}{dq_i} q_i + p_i = p_i (1 - \frac{1}{σ(c_i)})$
  \item To optimize profit $π_i = p_i q_i - wf - \frac{w}{φ} q_i$, set $MR_i = MC$
  \item We obtain the Profit max condition (PP): $\frac{p_i}{w} = \frac{σ(c_i)}{σ(c_i) - 1} \frac{1}{φ}$
  \item (Average Cost is $AC_i = f/q_i + w / φ$)
\end{itemize}

\subsection{Solving}
\begin{itemize}
  \item From $π_i = 0$, we obtain the Free entry condition (ZZ): $\frac{p_i}{w} = \frac{f}{L c_i} + \frac{1}{φ}$
  \item $L = \sum_{1 ≤ i ≤ n} l_i = nf + \frac{\sum_{1 ≤ i ≤ n} q_i}{φ}$
  \item By symmetry, $c_i = c$, $q_i = q$, $p_i = p$
  \item $L = nf + \frac{nLc}{φ}$ thus $n = \frac{L}{f + \frac{Lc}{φ}} = \frac{1}{\frac{f}{L} + \frac{c}{φ}}$
\end{itemize}

\subsection[In the (c, p/w) space]{In the $(c, p/w)$ space}
ZZ curve:
\begin{itemize}
  \item strictly decreasing
  \item higher consumption, higher production, lower average costs, lower $p/w$
  \item with $c$ constant, bigger $L$, more sales, lower $p/w$
\end{itemize}
PP curve:
\begin{itemize}
  \item Flat iff $σ'(c) = 0$
  \item Strictly increasing if $σ'(c) < 0$ ($\frac{σ(c_i)}{σ(c_i) - 1} = 1 + \frac{1}{σ(c_i) - 1}$, $\frac{1}{σ(c_i) - 1}$ composes two decreasing functions)
  \item $σ'(c) < 0$ behaviorally reasonable: bigger consumers are richer thus less sensitive to prices
  \item Higher consumption, less elastic demand, higher markup, more market power, higher prices
\end{itemize}

\subsection{Doubling market size}
Let’s double $L$
\begin{itemize}
  \item ZZ curve shifts downwards
  \item $c^{(1)} < c^{(0)}$
  \item More varieties: $n^{(0)} = \frac{1}{\frac{f}{L} + \frac{c^{(0)}}{φ}} < \frac{1}{\frac{f}{2L} + \frac{c^{(1)}}{φ}} = n^{(1)}$
  \item Under assumption $σ'(c) < 0$, $σ(c^{(0)}) < σ(c^{(1)})$ thus $(\frac{p}{w})^{(1)} < (\frac{p}{w})^{(0)}$
  \item Under assumption $σ'(c) < 0$, some firms exit: $n^{(1)} = \frac{1}{\frac{f}{2L} + \frac{c^{(1)}}{φ}} < \frac{1}{\frac{f}{2L} + \frac{c^{(0)}}{2 φ}} = 2 n^{(0)}$, equivalently, $c^{(0)} < 2 c^{(1)}$.
  Proof: let $(c^{(1)}, (\frac{p}{w})^{(1)})$ be the equilibrium after doubling $L$, thus $(2c^{(1)}, (\frac{p}{w})^{(1)})$ is on the ZZ curve before doubling $L$, and all points with $c ≥ 2 c^{(1)}$ are strictly below the PP curve (because its derivative is positive) and not below the ZZ curve (because its derivative is negative) so not equilibrium points.
\end{itemize}

\section{CES utility}
\subsection{Consumer}
\begin{itemize}
  \item A single consumer utility is $U(\bar{c}) = (\sum_{1 ≤ i ≤ n} c_i^\frac{σ - 1}{σ})^\frac{σ}{σ - 1}$ (a country $j$ utility is $U_j(L\bar{c})) = L U(\bar{c})$)
  \item Revenue constraint is $w = \sum p_i c_i$
  \item From the Lagrangian we obtain $\frac{σ}{σ - 1} U(\bar{c})^\frac{1}{σ} \frac{σ - 1}{σ} c_i^{\frac{σ - 1}{σ} - 1} - λ p_i = 0$ (because $\frac{σ}{σ - 1} - 1 = \frac{σ}{σ - 1}\frac{1}{σ}$), thus $c_i = λ^{-σ} p_i^{-σ} U(\bar{c})$
  \item Thus $U(\bar{c}) = (\sum p_i^{1 - σ})^{\frac{-σ}{1 - σ}} λ^{-σ} U(\bar{c})$, whence $λ^σ = (\sum p_i^{1 - σ})^{\frac{-σ}{1 - σ}}$ and $c_i = \frac{p_i^{-σ} U(\bar{c})}{(\sum p_i^{1 - σ})^{\frac{-σ}{1 - σ}}}$
  \item Thus $p_i c_i = \frac{p_i^{1 - σ} U(\bar{c})}{(\sum p_i^{1 - σ})^{\frac{-σ}{1 - σ}}}$, and summing over $i$, $w = \frac{\sum p_i^{1 - σ}}{(\sum p_i^{1 - σ})^{\frac{-σ}{1 - σ}}} U(\bar{c}) = (\sum p_i^{1 - σ})^{\frac{1}{1 - σ}} U(\bar{c})$
  \item Define the ideal price index as $P = \frac{w}{U(\bar{c})} = (\sum p_i^{1 - σ})^\frac{1}{1 - σ}$: it increases in the same way that the welfare decreases (price-index elasticity of utility is minus one); it is the cost of one unit of happiness
  \item It follows that $c_i = (\frac{p_i}{P})^{-σ} \frac{w}{P}$, and finally, using $q_i = Lc_i$: $q_i = (\frac{p_i}{P})^{-σ} \frac{w L}{P}$
  \item We obtain elasticity of demand $ε_i = \frac{∂q_i}{∂p_i} \frac{p_i}{q_i} = \left(-σ q_i (\frac{p_i}{P})^{-1} \frac{∂}{∂p_i} (\frac{p_i}{P})\right) \frac{p_i}{q_i}$, and, approximating $P$ as constant with respect to $p_i$ (how?), $ε_i ≈ -σ$
  \item Given $i ≠ j$, elasticity of substitution $ε_{ij} = \frac{∂ln(q_i / q_j)}{∂ln(p_j / p_i)} = -σ$ (using $\frac{q_i}{q_j} = (\frac{p_i}{p_j})^{-σ}$)
  \item Using symmetric prices, $P = n^{\frac{1}{1 - σ}} p$
  \item We have $U_j(\bar{q}) = \frac{Lw}{p} n^\frac{1}{σ - 1}$ thus $\frac{∂ \log U}{∂ \log n} = \frac{1}{σ - 1}$
  \item When varieties increase, welfare increases at rate $\frac{∂ \log U}{∂ \log n} = \frac{1}{σ - 1}$
\end{itemize}

\subsection{Producer}
\begin{itemize}
  \item As above, $l_i = f + q_i / φ$; to optimize profit set $MR_i = \frac{∂p_i}{∂q_i}q_i + p_i = \frac{p_i}{ε_i} + p_i = MC = \frac{w}{φ}$, obtain $\frac{p_i}{w} = \frac{σ}{σ - 1} \frac{1}{φ}$
  \item At zero profit $\frac{π_i}{w} = \frac{p_i}{w} q_i - f - \frac{1}{φ} q_i = 0$, we obtain a constant output per variety, $q_i = φ f (σ - 1)$
\end{itemize}

\subsection{Solving, in autarky}
\begin{itemize}
  \item Using symmetric goods and market clearing condition $L = nf + n \frac{q}{φ}$, we obtain $n = \frac{L}{σf}$
  \item Thus, $P = \frac{1}{n^{\frac{1}{σ - 1}}} \frac{σ}{σ - 1} \frac{w}{φ} = (\frac{σf}{L})^{\frac{1}{σ - 1}} \frac{σ}{σ - 1} \frac{w}{φ}$
  \item We see that increasing $n$, or increasing $L$, decreases the price index, thus increases welfare
\end{itemize}

\subsection{Iceberg trading}
Country $D$ (with variable levels denoted as $v_D^{(0)}$) opens up and starts trading with country $X$ (with variable levels denoted as $v_X^{(0)}$); resulting variable levels denoted as $v_D^{(1)}$
\begin{itemize}
  \item Iceberg type trading cost: domestic varieties have price $p_{D → X} = p_{D → X}^{(1)} = τ p_D$ abroad, with $1 ≤ τ$, $p_D = p_D^{(0)} = p_D^{(1)} = \frac{σ}{σ - 1} \frac{w}{φ}$
  \item Total production of $D$ becomes $q_D^{(1)} = q_{D → D} + τ q_{D → X}$, with $q_{D → X}$ the quantity that effectively arrives abroad
  \item Thus we sell $q_D^{(1)}$ at $p_D$, equivalently, $q_D$ at $p_D$ domestically and $q_{D → X}$ at $p_{D → X}$ abroad
  \item Total profit is $π^{(1)} = p_D q_D^{(1)} - w (f + \frac{q_D^{(1)}}{φ}) = \frac{1}{σ - 1} \frac{w q_D^{(1)}}{φ} - w f$
  \item From $π = 0$ we obtain $q_D^{(1)} = φ f (σ - 1) = q_D^{(0)}$
  \item Using $L_D = n_D^{(1)} f + n_D^{(1)} \frac{q_D^{(1)}}{φ}$, we obtain $n_D^{(1)} = \frac{L_D}{σ f} = n_D^{(0)}$
  \item From above, $q_{D → X} = \left(\frac{p_{D → X}}{P_X}\right)^{-σ} \frac{w_X L_X}{P_X}$, thus $q_{D → X} = \left(\frac{τ p_D}{P_X}\right)^{-σ} \frac{w_X L_X}{P_X}$
  \item It follows that the value $X = τ p_D q_{D → X} n$ of aggregate exports (the gravity equation) depends linearly on $L_D L_X$ and on $τ^{1 - σ}$ (also, $0 ≤ τ^{1 - σ} ≤ 1$)
\end{itemize}
Mechanisms through which trade increases:
\begin{description}
  \item[$τ$ finite] Trade increases through extensive margins: countries start exchanging every varieties
  \item[$τ$ reduces] Trade increases through intensive margins: higher export in quantity but constant number of varieties
\end{description}
Effect of trade:
\begin{itemize}
  \item ${P_D}^{(1)} = [n_D {p_D}^{1 - σ} + n_X {(τ p_D)}^{1 - σ}]^\frac{1}{1 - σ}$
  \item With $τ = 1$ and $n_X = n_D$, we get ${P_D}^{(1)} = \frac{{P_D}^{(0)}}{2^\frac{1}{σ - 1}}$: price index decreases
  \item No pro-competitive effect: $σ$ fixed, does not contribute to lowering prices
  \item From trade balance $X_D = X_X$
\end{itemize}

Problem is that more productive firms seem to export, which the model does not say; this is not solvable even with different trading costs (not sure why)

TODO solve the exercice CES demand with vertical differentiation

\section{Melitz}
Here we implicitly normalize by setting $w = 1$.

We write $φ_d$ the equivalent of $φ_a$ under non-autarky so actually $φ_a^{(1)}$ as compared to $φ_a^{(0)}$

Note that $π(φ)$ is the profit of a single firm of the type $φ$; to get the profit of the whole type $φ$ we’d need to consider the density $g$.

Note that there is no aggregate profit to redistribute because the total profit is precisely the total entry cost (zero entry condition).

%\bibliography{bibl}

\end{document}

